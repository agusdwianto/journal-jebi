\documentclass[final,5p,times,twocolumn]{elsarticle}

% =========================================================
%  JEBI / Elsevier-style Manuscript Template (APA 7)
%  Engine recommended: XeLaTeX (works with your CI)
% =========================================================

% ---------- Language & Typography ----------
\usepackage[english]{babel}      % Elsevier/JBR normally English
\usepackage{microtype}          % better justification
\usepackage{csquotes}           % recommended for APA-ish quotes

% ---------- Layout ----------
\usepackage{geometry}
\geometry{left=1.6cm,right=1.6cm,top=2.0cm,bottom=2.0cm}

% ---------- Graphics ----------
\usepackage{graphicx}
\usepackage{cuted}              % full-width strip for branding header
\usepackage{xcolor}

% ---------- Tables ----------
\usepackage{booktabs}
\usepackage{threeparttable}
\usepackage{array}

% ---------- Links ----------
\usepackage{hyperref}
\hypersetup{
  colorlinks=true,
  linkcolor=blue,
  citecolor=blue,
  urlcolor=blue,
  pdftitle={Journal Economic Business Innovation (JEBI)},
  pdfauthor={JEBI Authors}
}

% ---------- Citations: APA 7 (Author–Year) ----------
\usepackage[authoryear,round]{natbib}
\biboptions{authoryear,round,semicolon}

% Use an Elsevier harvard-like style (closest in elsarticle family).
% For strict APA 7 formatting you normally need biblatex-apa, but
% Elsevier workflows commonly use natbib + elsarticle-harv.
\bibliographystyle{elsarticle-harv}

% ---------- Journal Identity ----------
\journal{Journal Economic Business Innovation (JEBI)}

% ---------- Small helpers ----------
\newcommand{\artinfoTitle}[1]{\textbf{\sffamily\small #1}}
\newcommand{\tight}{\setlength{\parskip}{0pt}\setlength{\itemsep}{2pt}}

\begin{document}

% =========================================================
%  TOP BRANDING STRIP (optional - uses your images)
%  Make sure these files exist:
%   logo_jebi.jpg, homepage_jebi.jpg, cover_jebi.jpg
% =========================================================
\begin{strip}
  \vspace{-1.3cm}
  \noindent
  \begin{minipage}[c]{0.18\textwidth}
    \includegraphics[width=\textwidth]{logo_jebi.jpg}
  \end{minipage}\hfill
  \begin{minipage}[c]{0.64\textwidth}
    \centering
    \includegraphics[width=0.92\textwidth]{homepage_jebi.jpg}
  \end{minipage}\hfill
  \begin{minipage}[c]{0.14\textwidth}
    \raggedleft
    \includegraphics[width=\textwidth]{cover_jebi.jpg}
  \end{minipage}
  \vspace{6pt}
  \hrule height 0.8pt
  \vspace{10pt}
\end{strip}

% =========================================================
%  FRONTMATTER (Elsevier standard)
% =========================================================
\begin{frontmatter}

\title{Digital transformation as a multi-phase process: A strategic framework for emerging markets}

% --- Authors (example) ---
\author[inst1]{First Author Name\corref{cor1}}
\ead{first.author@domain.com}

\author[inst2]{Second Author Name}
\ead{second.author@domain.com}

\cortext[cor1]{Corresponding author.}

\address[inst1]{PT. Inovasi Analisis Data, Semarang 50266, Indonesia}
\address[inst2]{Department of Economics, University Name, Country}

% --- Abstract (keep concise like JBR) ---
\begin{abstract}
This study investigates how digital transformation unfolds over time and how organizations align strategy, structure, and execution. Using a (method) approach with (data/source), we find that (key results). The findings contribute to (theory/practice) by (main contribution) and offer implications for (managers/policymakers).
\end{abstract}

% --- Keywords (JBR style: short, specific) ---
\begin{keyword}
Digital transformation \sep Corporate strategy \sep Organizational adaptation \sep Business model innovation \sep Emerging markets
\end{keyword}

\end{frontmatter}

% =========================================================
%  FIRST-PAGE "ARTICLE INFO" + "ABSTRACT" BLOCK (optional)
%  If you want it closer to JBR screenshot: show a boxed block.
%  NOTE: Frontmatter already has Abstract+Keywords; this is a VISUAL add-on.
% =========================================================
\vspace{-2mm}
\noindent\rule{\columnwidth}{0.4pt}
\vspace{2mm}

\noindent
\begin{minipage}[t]{0.48\columnwidth}
  \artinfoTitle{ARTICLE INFO}\par
  \vspace{4pt}
  \textit{Keywords:}\par
  Digital transformation\par
  Corporate strategy\par
  Organizational tension\par
  Structural adjustment\par
  Organizational adaptation\par
  \vspace{6pt}
  \textit{JEL classification (optional):}\par
  O33; L25; M10
\end{minipage}
\hfill
\begin{minipage}[t]{0.48\columnwidth}
  \artinfoTitle{ABSTRACT}\par
  \vspace{4pt}
  This study investigates how digital transformation unfolds over time within organizations. Drawing on (design), we trace (process) and identify (patterns). The study advances understanding of (topic) by showing (core insight) and offers implications for (practice).
\end{minipage}

\vspace{2mm}
\noindent\rule{\columnwidth}{0.4pt}
\vspace{3mm}

% =========================================================
%  MAIN MANUSCRIPT
% =========================================================

\section{Introduction}
Write your introduction here. Cite with APA author--year style, e.g.,
\citet{Vial2019} argues that digital transformation is a multi-dimensional phenomenon.
Other studies report similar patterns \citep{Verhoef2021}.

\section{Literature review}
\subsection{Digital transformation}
Keep paragraphs tight, define constructs, show gap.

\subsection{Business model innovation}
Explain theory and what your paper adds.

\section{Methodology}
Describe design, sample/data, measures, analysis steps.

\section{Results}
Present results with tables/figures.

\begin{table}[t]
\centering
\caption{Example table (professional style)}
\begin{threeparttable}
\begin{tabular}{lcc}
\toprule
\textbf{Variable} & \textbf{Coefficient} & \textbf{$p$-value} \\
\midrule
Digital adoption & 0.45 & < .05 \\
Market agility   & 0.32 & < .01 \\
\bottomrule
\end{tabular}
\begin{tablenotes}\small
\item Notes: Report robust standard errors if applicable.
\end{tablenotes}
\end{threeparttable}
\end{table}

\section{Discussion}
Interpret findings, compare with literature, theoretical and managerial implications.

\section{Conclusion}
Summarize contribution, limitations, and future research.

% =========================================================
%  REFERENCES
%  Use references.bib (BibTeX). Example keys: Vial2019, Verhoef2021, etc.
% =========================================================
\bibliography{references}

\end{document}
