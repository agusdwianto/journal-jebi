\documentclass[final,5p,times,twocolumn]{elsarticle}
\usepackage[utf8]{inputenc}
\usepackage{graphicx}
\usepackage{geometry}

% Mengatur margin agar pas dengan header gambar
\geometry{left=1.5cm,right=1.5cm,top=2cm,bottom=2cm}

\journal{Journal Economic Business Innovation (JEBI)}

\begin{document}

%%% --- HEADER CUSTOM JEBI --- %%%
\begin{strip}
\begin{minipage}[t]{0.2\textwidth}
    \includegraphics[width=\textwidth]{logo_jebi.jpg} % Pojok Kiri
\end{minipage}
\begin{minipage}[t]{0.6\textwidth}
    \centering
    \includegraphics[width=0.8\textwidth]{homepage_jebi.jpg} % Tengah
\end{minipage}
\begin{minipage}[t]{0.2\textwidth}
    \raggedleft
    \includegraphics[width=0.8\textwidth]{cover_jebi.jpg} % Pojok Kanan
\end{minipage}
\vspace{5pt}
\hrule
\end{strip}
%%% -------------------------- %%%

\begin{frontmatter}

\title{Judul Artikel Jurnal JEBI Anda}

\author{Nama Penulis Pertama\corref{cor1}}
\ead{email@analisisdata.co.id}
\author{Nama Penulis Kedua}

\address{Inovasi Analisis Data, Semarang, Indonesia}

\begin{abstract}
Abstrak ini akan tampil dalam format dua kolom yang rapi di bawah header khusus JEBI yang baru saja kita buat. Format ini mengikuti standar publikasi internasional.
\end{abstract}

\begin{keyword}
Inovasi \sep Bisnis \sep Ekonomi \sep JEBI
\end{keyword}

\end{frontmatter}

\section{Introduction}
Teks pendahuluan Anda di sini. Robot GitHub akan otomatis mencetak logo dan cover di bagian atas halaman pertama.

\end{document}
