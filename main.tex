\documentclass[final,5p,times,twocolumn]{elsarticle}

% =========================================================
% Elsevier (JBR-like) two-column + APA 7 (author–year)
% XeLaTeX friendly
% =========================================================

% --- Fonts (XeLaTeX) ---
\usepackage{fontspec}
\setmainfont{TeX Gyre Termes} % Times-like, aman di CI

% --- Language & typography ---
\usepackage[english]{babel}
\usepackage{microtype}

% --- Layout ---
\usepackage{geometry}
\geometry{a4paper,left=1.8cm,right=1.8cm,top=2.0cm,bottom=2.0cm}

% --- Graphics / tables ---
\usepackage{graphicx}
\usepackage{booktabs}

% --- Full-width blocks in two-column ---
\usepackage{cuted}

% --- Boxes (for ARTICLE INFO / ABSTRACT like JBR) ---
\usepackage{tcolorbox}
\tcbuselibrary{breakable}
\newtcolorbox{jbrbox}{
  enhanced,
  breakable,
  colback=white,
  colframe=black!35,
  boxrule=0.4pt,
  arc=0pt,
  left=8pt,right=8pt,top=8pt,bottom=8pt
}

% --- Hyperlinks ---
\usepackage{hyperref}
\hypersetup{
  colorlinks=true,
  linkcolor=blue,
  citecolor=blue,
  urlcolor=blue
}

% --- APA 7 author–year citations (Elsevier-compatible) ---
\usepackage[authoryear]{natbib}

% Journal name (optional)
\journal{Journal Economic Business Innovation (JEBI)}

% =========================================================
% (OPTIONAL) Top banner like Elsevier/JBR
% You can replace images or text as needed
% =========================================================
\newcommand{\TopLineLeft}{Journal of Business Research 202 (2026) 115796} % contoh
\newcommand{\TopLineURL}{\href{https://www.elsevier.com/locate/jbusres}{www.elsevier.com/locate/jbusres}}

\begin{document}

% =========================================================
% TOP HEADER (JBR-like)
% =========================================================
\begin{strip}
  \vspace{-1.2cm}
  \noindent
  \centering
  {\small \TopLineLeft}\par
  \vspace{6pt}
  \hrule height 0.4pt
  \vspace{10pt}

  % Banner block
  \noindent
  \begin{minipage}[c]{0.18\textwidth}
    % Optional: Elsevier tree/logo placeholder (replace with your image if you have it)
    % \includegraphics[width=\textwidth]{elsevier_logo.png}
    \centering
    {\Large \textsc{ELSEVIER}}
  \end{minipage}%
  \hfill
  \begin{minipage}[c]{0.62\textwidth}
    \centering
    {\small Contents lists available at \textbf{ScienceDirect}}\par
    \vspace{4pt}
    {\Large \textbf{Journal of Business Research}}\par
    \vspace{4pt}
    {\small journal homepage:\ \TopLineURL}\par
  \end{minipage}%
  \hfill
  \begin{minipage}[c]{0.18\textwidth}
    \centering
    % Optional: journal cover placeholder
    % \includegraphics[width=\textwidth]{journal_cover.png}
    {\fbox{\parbox[c][2.2cm][c]{0.95\textwidth}{\centering Cover}}}
  \end{minipage}

  \vspace{10pt}
  \hrule height 0.8pt
  \vspace{10pt}
\end{strip}

% =========================================================
% FRONTMATTER (Elsevier standard)
% =========================================================
\begin{frontmatter}

\title{Digital transformation as a multi-phase process: a longitudinal study of corporate strategy and business unit adaptation}

% Authors (contoh)
\author[inst1]{First Author\corref{cor1}}
\author[inst2]{Second Author}
\author[inst1]{Third Author}

\cortext[cor1]{Corresponding author.}
\ead{first.author@email.com}

\address[inst1]{Department of Management, University Name, City, Country}
\address[inst2]{Department of Economics, University Name, City, Country}

% NOTE:
% Kita sengaja TIDAK pakai abstract default di sini,
% karena abstract akan kita tata ulang seperti JBR (dua kolom box) setelah frontmatter.

\end{frontmatter}

% =========================================================
% ARTICLE INFO (left) + ABSTRACT (right) in one full-width row
% =========================================================
\begin{strip}
  \noindent
  \begin{minipage}[t]{0.30\textwidth}
    \begin{jbrbox}
      {\small\textbf{ARTICLE INFO}}\par
      \vspace{6pt}
      {\small\textbf{Keywords:}}\par
      {\small Digital transformation}\par
      {\small Corporate strategy}\par
      {\small Organizational tension}\par
      {\small Structural adjustment}\par
      {\small Business unit adaptation}\par
      {\small Process perspective}\par
      \vspace{10pt}
      {\small\textbf{JEL classification:}}\par
      {\small M10; M15; O33}\par
      \vspace{10pt}
      {\small\textbf{Article history:}}\par
      {\small Received 26 February 2024}\par
      {\small Revised 14 October 2025}\par
      {\small Accepted 16 October 2025}\par
      {\small Available online 27 October 2025}\par
    \end{jbrbox}
  \end{minipage}
  \hfill
  \begin{minipage}[t]{0.66\textwidth}
    \begin{jbrbox}
      {\small\textbf{ABSTRACT}}\par
      \vspace{6pt}
      \small
      This study investigates how digital transformation unfolds over time within a multi-business manufacturing firm.
      Drawing on a longitudinal case study, we trace the dynamics of digital transformation across empirically derived phases:
      experimentation, consolidation, and acceleration. We identify recurring tensions between corporate strategy and business unit adaptation,
      and show how structural adjustments, capability building, and organizational alignment shape outcomes.
      The findings advance process-oriented perspectives on strategy by demonstrating how recursive patterns of tension,
      structural change, and organizational adaptation drive digital transformation in complex, multi-level firms.
    \end{jbrbox}
  \end{minipage}

  \vspace{10pt}
\end{strip}

% =========================================================
% MAIN TEXT (two-column)
% =========================================================
\section{Introduction}
Emerging digital technologies create disruptive challenges and opportunities for firms.
Prior studies emphasize that digital transformation often unfolds as a multi-phase process \citep{Vial2019, Warner2019}.
According to \citet{Hanelt2021}, transformation initiatives frequently involve tensions between legacy structures and new capabilities.

\section{Literature review}
Write your literature review here. Use APA author–year citations:
\begin{itemize}
  \item Narrative: \citet{Vial2019} argues that \dots
  \item Parenthetical: \citep{Warner2019, Hanelt2021}
\end{itemize}

\section{Methodology}
Describe the research design, data, and analysis.

\section{Results}
Present results with figures and tables.

\section{Discussion}
Interpret results and connect to literature.

\section{Conclusion}
Summarize contributions and implications.

\section*{Declaration of competing interest}
The authors declare no competing interests.

\section*{Acknowledgements}
(Optional)

\section*{Funding}
(Optional)

% =========================================================
% REFERENCES — APA-like author–year (Elsevier Harvard)
% =========================================================
\bibliographystyle{elsarticle-harv}
\bibliography{references}

\end{document}
