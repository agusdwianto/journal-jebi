%% OPSI PENTING:
%% 5p = Format 2 kolom penuh (Final version)
%% times = Font Times New Roman standar Elsevier
\documentclass[final,5p,times,twocolumn]{elsarticle}

%% --- PAKET STANDAR ---
\usepackage[utf8]{inputenc}
\usepackage[T1]{fontenc}
\usepackage[english]{babel}
\usepackage{amssymb}
\usepackage{amsmath}
\usepackage{booktabs} % Untuk tabel garis rapi
\usepackage{graphicx}
\usepackage{xcolor}

%% --- PAKET HEADER KUSTOM (KUNCI AGAR 3 GAMBAR RAPI) ---
\usepackage{fancyhdr}
\usepackage{geometry}

%% Atur margin agar ada ruang khusus di atas untuk 3 gambar
%% 'top=3.5cm' memberikan ruang yang cukup agar gambar tidak menabrak judul
\geometry{
    top=3.5cm,      
    bottom=2.5cm, 
    left=1.5cm, 
    right=1.5cm,
    headheight=2.5cm, % Tinggi area khusus gambar header
    includehead
}

%% --- DEFINISI HEADER HALAMAN PERTAMA (3 GAMBAR WAJIB) ---
\fancypagestyle{firstpage}{
    \fancyhf{} % Bersihkan header default
    \renewcommand{\headrulewidth}{1pt} % Garis tipis di bawah gambar (opsional, bisa dihapus)
    
    %% Posisi Gambar: Kiri (Logo), Tengah (Homepage), Kanan (Cover)
    %% Gunakan 'keepaspectratio' agar gambar tidak gepeng
    %% Menggunakan 'example-image' jika file tidak ditemukan
    \fancyhead[L]{\IfFileExists{logo_jebi.jpg}{\includegraphics[height=1.5cm, keepaspectratio]{logo_jebi.jpg}}{\includegraphics[height=1.5cm]{example-image-a}}} 
    \fancyhead[C]{\IfFileExists{homepage_jebi.jpg}{\includegraphics[height=1.5cm, keepaspectratio]{homepage_jebi.jpg}}{\includegraphics[height=1.5cm]{example-image-b}}}
    \fancyhead[R]{\IfFileExists{cover_jebi.jpg}{\includegraphics[height=1.5cm, keepaspectratio]{cover_jebi.jpg}}{\includegraphics[height=1.5cm]{example-image-c}}}
}

%% --- HEADER HALAMAN SELANJUTNYA (NAMA JURNAL SAJA) ---
%% Halaman 2 dst tidak perlu gambar besar, cukup teks nama jurnal agar enak dibaca
\fancypagestyle{standardpage}{
    \fancyhf{}
    \fancyhead[R]{\footnotesize \itshape Journal Economic Business Innovation (JEBI)}
    \fancyhead[L]{\footnotesize \itshape Vol. 1, No. 1, 2026}
    \fancyfoot[C]{\thepage} % Nomor halaman otomatis di tengah bawah
}

%% --- SETTING LINK BIRU (STANDAR SCIENCEDIRECT) ---
\usepackage{hyperref}
\hypersetup{
    colorlinks=true,
    linkcolor=blue,
    citecolor=blue,
    urlcolor=blue,
    pdftitle={JEBI Article}
}

%% --- HILANGKAN TULISAN "Preprint submitted to..." ---
\makeatletter
\def\ps@pprintTitle{%
     \let\@oddhead\@empty
     \let\@evenhead\@empty
     \def\@oddfoot{\footnotesize\itshape
       Journal Economic Business Innovation (JEBI) \hfill \today}%
     \let\@evenfoot\@oddfoot}
\makeatother

\journal{Journal Economic Business Innovation (JEBI)}

\begin{document}

%% --- PENGATURAN NOMOR HALAMAN ---
%% Ubah angka 1 di bawah ini jika ingin mulai dari halaman lain (misal: 455)
\setcounter{page}{1}

%% --- AKTIFKAN HEADER 3 GAMBAR KHUSUS HALAMAN 1 ---
\thispagestyle{firstpage}
\pagestyle{standardpage} % Halaman 2 dst pakai header teks biasa

\begin{frontmatter}

%% --- JUDUL & PENULIS (TETAP RAPI DI BAWAH GAMBAR) ---
\title{Digital Transformation in Emerging Markets: A Strategic Framework for Sustainable Economic Growth}

\author[1]{First Author Name\corref{cor1}}
\ead{author@analysisdata.co.id}
\author[2]{Second Author Name}

\cortext[cor1]{Corresponding author.}
\address[1]{PT. Inovasi Analisis Data, Semarang 50266, Indonesia}
\address[2]{Faculty of Economics, University Name, Country}

%% --- ABSTRAK (Format Kotak Rapi) ---
\begin{abstract}
\textbf{Purpose:} This study investigates the impact of digital transformation strategies on economic sustainability in emerging markets.
\textbf{Design/methodology/approach:} Using a quantitative approach, data were collected from 150 geographically diverse samples and analyzed using structural equation modeling (SEM).
\textbf{Findings:} The results indicate a significant positive correlation between digital infrastructure investment and long-term business resilience.
\textbf{Originality/value:} This research provides a novel framework for policymakers in developing regions to integrate technology into traditional economic sectors.
\end{abstract}

%% --- KATA KUNCI ---
\begin{keyword}
Digital Economy \sep Business Strategy \sep Innovation \sep Sustainability \sep JEBI
\end{keyword}

\end{frontmatter}

%% === MULAI ISI ARTIKEL (2 KOLOM OTOMATIS) ===

\section{Introduction}
The rapid evolution of digital technology has fundamentally altered the global economic landscape. In the context of emerging markets, this transformation presents both significant challenges and unprecedented opportunities. As noted in recent studies, the integration of digital tools is essential for modern business survival \cite{JEBI2026}.

This paper explores the intersection of economic policy and business innovation. By leveraging the framework established by the \textit{Journal Economic Business Innovation} (JEBI), we aim to provide empirical evidence supporting digital adoption.

\section{Literature Review}
\subsection{Theories of Innovation}
Innovation theory suggests that technological advancement is the primary driver of economic growth. Previous research has demonstrated that companies adopting agile methodologies outperform their traditional counterparts.

\subsection{Business Resilience}
Resilience is defined as the capacity to recover quickly from difficulties. In a business context, this involves the ability to pivot strategies in response to market disruptions.

\section{Methodology}
This study employs a descriptive quantitative method. Data collection was conducted through structured surveys distributed to industry leaders in Semarang and surrounding areas.

\section{Results and Discussion}
The analysis reveals that 75\% of respondents believe digital innovation is critical. 

%% CONTOH TABEL RAPI
\begin{table}[h]
    \centering
    \caption{Impact of Innovation on Revenue}
    \begin{tabular}{l c c}
        \toprule
        \textbf{Variable} & \textbf{Coefficient} & \textbf{P-Value} \\
        \midrule
        Digital Adoption & 0.45 & < 0.05 \\
        Market Agility   & 0.32 & < 0.01 \\
        \bottomrule
    \end{tabular}
\end{table}

\section{Conclusion}
In conclusion, this study confirms that digital transformation is not merely a trend but a necessity for economic sustainability.

%% --- DAFTAR PUSTAKA ---
\bibliographystyle{elsarticle-num} 
\bibliography{references}

\end{document}
