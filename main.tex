\documentclass[final,5p,times,twocolumn]{elsarticle}

% =========================================================
% CLEAN FINAL — Elsevier/JBR-like layout + APA 7 (author–year)
% Engine: XeLaTeX (recommended)
% =========================================================

% ---------- Language & Typography ----------
\usepackage[english]{babel}
\usepackage{microtype}

% ---------- Layout ----------
\usepackage{geometry}
\geometry{a4paper,left=1.6cm,right=1.6cm,top=2.0cm,bottom=2.0cm}

% ---------- Graphics / Tables ----------
\usepackage{graphicx}
\usepackage{booktabs}
\usepackage{threeparttable}
\usepackage{cuted} % full-width strip for top banner in two-column

% ---------- Links ----------
\usepackage{hyperref}
\hypersetup{
  colorlinks=true,
  linkcolor=blue,
  citecolor=blue,
  urlcolor=blue,
  pdftitle={Journal Economic Business Innovation (JEBI)},
  pdfauthor={JEBI Authors}
}

% ---------- APA 7 author–year citations (Elsevier-compatible) ----------
\usepackage[authoryear,round]{natbib}
\biboptions{authoryear,round,semicolon}
\bibliographystyle{elsarticle-harv}

% ---------- Journal Identity ----------
\journal{Journal Economic Business Innovation (JEBI)}

% ---------- Optional top line like JBR (edit as needed) ----------
\newcommand{\TopLineLeft}{Journal of Business Research 202 (2026) 115796}
\newcommand{\TopLineURL}{\href{https://www.elsevier.com/locate/jbusres}{www.elsevier.com/locate/jbusres}}

\begin{document}

% =========================================================
% TOP BANNER (JBR-like) — uses your existing images
% Files expected:
%   logo_jebi.jpg, homepage_jebi.jpg, cover_jebi.jpg
% =========================================================
\begin{strip}
  \vspace{-1.2cm}
  \noindent\centering
  {\small \TopLineLeft}\par
  \vspace{6pt}
  \hrule height 0.4pt
  \vspace{10pt}

  \noindent
  \begin{minipage}[c]{0.18\textwidth}
    \includegraphics[width=\textwidth]{logo_jebi.jpg}
  \end{minipage}\hfill
  \begin{minipage}[c]{0.64\textwidth}
    \centering
    {\small Contents lists available at \textbf{ScienceDirect}}\par
    \vspace{4pt}
    {\Large \textbf{Journal of Business Research}}\par
    \vspace{4pt}
    {\small journal homepage:\ \TopLineURL}\par
    \vspace{6pt}
    \includegraphics[width=0.92\textwidth]{homepage_jebi.jpg}
  \end{minipage}\hfill
  \begin{minipage}[c]{0.14\textwidth}
    \raggedleft
    \includegraphics[width=\textwidth]{cover_jebi.jpg}
  \end{minipage}

  \vspace{10pt}
  \hrule height 0.8pt
  \vspace{10pt}
\end{strip}

% =========================================================
% FRONTMATTER (Elsevier standard)
% =========================================================
\begin{frontmatter}

\title{Digital transformation as a multi-phase process: A longitudinal study of corporate strategy and business unit adaptation}

\author[inst1]{First Author Name\corref{cor1}}
\ead{first.author@domain.com}

\author[inst2]{Second Author Name}
\ead{second.author@domain.com}

\cortext[cor1]{Corresponding author.}

\address[inst1]{PT. Inovasi Analisis Data, Semarang 50266, Indonesia}
\address[inst2]{Department of Management, University Name, City, Country}

\begin{abstract}
This study investigates how digital transformation unfolds over time within organizations. Drawing on a longitudinal design, we trace key phases of experimentation, consolidation, and acceleration. We identify recurring tensions between corporate strategy and business unit adaptation and show how structural adjustments and capability building shape outcomes. The findings advance process-oriented perspectives on strategy and provide implications for managers and policymakers.
\end{abstract}

\begin{keyword}
Digital transformation \sep Corporate strategy \sep Organizational adaptation \sep Business unit adaptation \sep Process perspective
\end{keyword}

\end{frontmatter}

% =========================================================
% JBR-LIKE "ARTICLE INFO" (left) + "ABSTRACT" (right) row
% (visual block; keep short)
% =========================================================
\begin{strip}
  \noindent\rule{\textwidth}{0.4pt}
  \vspace{6pt}

  \noindent
  \begin{minipage}[t]{0.30\textwidth}
    \textbf{\small ARTICLE INFO}\par
    \vspace{4pt}
    \small
    \textit{Keywords:}\par
    Digital transformation\par
    Corporate strategy\par
    Organizational tension\par
    Structural adjustment\par
    Business unit adaptation\par
    Process perspective\par
    \vspace{8pt}
    \textit{Article history:}\par
    Received DD Month YYYY\par
    Revised DD Month YYYY\par
    Accepted DD Month YYYY\par
    Available online DD Month YYYY\par
  \end{minipage}
  \hfill
  \begin{minipage}[t]{0.66\textwidth}
    \textbf{\small ABSTRACT}\par
    \vspace{4pt}
    \small
    This study examines how digital transformation unfolds over time and how organizations align strategy, structure, and execution. Using (method) and (data/source), we find that (key results). The study contributes by (main contribution) and offers implications for (practice/policy).
  \end{minipage}

  \vspace{6pt}
  \noindent\rule{\textwidth}{0.4pt}
  \vspace{10pt}
\end{strip}

% =========================================================
% MAIN MANUSCRIPT (Two-column)
% =========================================================
\section{Introduction}
Emerging digital technologies create disruptive challenges and opportunities for firms.
Prior studies emphasize that digital transformation unfolds as a multi-phase process \citep{Vial2019,Warner2019}.
According to \citet{Hanelt2021}, transformation initiatives often involve tensions between legacy structures and new capabilities.

\section{Literature review}
Use APA author--year citations:
\begin{itemize}
  \item Narrative: \citet{Vial2019} argues that \dots
  \item Parenthetical: \citep{Warner2019,Hanelt2021}
\end{itemize}

\section{Methodology}
Describe the research design, data, and analysis.

\section{Results}
Present results with tables/figures.

\begin{table}[t]
\centering
\caption{Example table (professional style)}
\begin{threeparttable}
\begin{tabular}{lcc}
\toprule
\textbf{Variable} & \textbf{Coefficient} & \textbf{$p$-value} \\
\midrule
Digital adoption & 0.45 & < .05 \\
Market agility   & 0.32 & < .01 \\
\bottomrule
\end{tabular}
\begin{tablenotes}\small
\item Notes: Report robust standard errors if applicable.
\end{tablenotes}
\end{threeparttable}
\end{table}

\section{Discussion}
Interpret results, compare with literature, and outline implications.

\section{Conclusion}
Summarize contributions, limitations, and future research directions.

\section*{Declaration of competing interest}
The authors declare no competing interests.

\section*{Acknowledgements}
(Optional)

\section*{Funding}
(Optional)

% =========================================================
% REFERENCES (APA-like author–year)
% =========================================================
\bibliography{references}

\end{document}
