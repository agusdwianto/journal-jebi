\documentclass[final,5p,times,twocolumn]{elsarticle}

% =========================================================
%  TEMPLATE "ANTI-GAGAL" JEBI / ELSEVIER
%  Gunakan ini untuk tes build pertama kali.
% =========================================================

% 1. Paket Standar
\usepackage[english]{babel}
\usepackage{microtype}
\usepackage{geometry}
\geometry{a4paper,left=1.6cm,right=1.6cm,top=2.0cm,bottom=2.0cm}

% 2. Grafik & Tabel
\usepackage{graphicx}
\usepackage{booktabs}
\usepackage{cuted} % Untuk header lebar penuh (strip)

% 3. Link PDF
\usepackage{hyperref}
\hypersetup{colorlinks=true,linkcolor=blue,citecolor=blue,urlcolor=blue}

% 4. Sitasi (APA Style)
\usepackage[authoryear,round]{natbib}
\bibliographystyle{elsarticle-harv}

% 5. Identitas Jurnal
\journal{Journal Economic Business Innovation (JEBI)}
\newcommand{\TopLineLeft}{Journal of Business Research 202 (2026) 115796}

\begin{document}

% =========================================================
%  HEADER / BRANDING (SAFE MODE)
%  Saya ganti gambar dengan KOTAK KOSONG (\framebox)
%  agar GitHub tidak error mencari file gambar yg belum ada.
% =========================================================
\begin{strip}
  \vspace{-1.5cm}
  \noindent\centering
  {\small \TopLineLeft}\par
  \vspace{6pt}
  \hrule height 0.4pt
  \vspace{10pt}

  \noindent
  % --- Logo Kiri ---
  \begin{minipage}[c]{0.18\textwidth}
    \centering
    \framebox[\textwidth][c]{\textbf{LOGO HERE}} 
    % Nanti jika file sudah ada, ganti baris atas dengan:
    % \includegraphics[width=\textwidth]{logo_jebi.jpg}
  \end{minipage}\hfill
  % --- Teks Tengah ---
  \begin{minipage}[c]{0.64\textwidth}
    \centering
    {\small Contents lists available at \textbf{ScienceDirect}}\par
    \vspace{4pt}
    {\Large \textbf{Journal of Business Research}}\par
    \vspace{4pt}
    \framebox[0.9\textwidth][c]{BANNER HOMEPAGE HERE}
  \end{minipage}\hfill
  % --- Cover Kanan ---
  \begin{minipage}[c]{0.14\textwidth}
    \raggedleft
    \framebox[\textwidth][c]{\textbf{COVER}}
  \end{minipage}

  \vspace{10pt}
  \hrule height 0.8pt
  \vspace{10pt}
\end{strip}

% =========================================================
%  JUDUL & PENULIS (Copy dari Word ke sini)
% =========================================================
\begin{frontmatter}

\title{Judul Artikel Dari File Word Penulis Masukkan Disini}

\author[1]{Nama Penulis Satu\corref{cor1}}
\ead{email@domain.com}
\author[2]{Nama Penulis Dua}

\cortext[cor1]{Corresponding author.}
\address[1]{Afiliasi Penulis Satu, Kota, Negara}
\address[2]{Afiliasi Penulis Dua, Kota, Negara}

% Abstract dikosongkan disini, kita pakai layout khusus di bawah
\end{frontmatter}

% =========================================================
%  ABSTRAK & INFO ARTIKEL (Gaya JBR)
% =========================================================
\begin{strip}
  \noindent\rule{\textwidth}{0.4pt}
  \vspace{6pt}
  \noindent
  \begin{minipage}[t]{0.30\textwidth}
    \textbf{\small ARTICLE INFO}\par
    \small
    \textit{Keywords:}\par
    Digital transformation\par
    Strategy\par
    Innovation\par
    \vspace{6pt}
    \textit{Article history:}\par
    Received 2026\par
    Accepted 2026\par
  \end{minipage}%
  \hfill
  \begin{minipage}[t]{0.66\textwidth}
    \textbf{\small ABSTRACT}\par
    \small
    % --- PASTE ABSTRAK DARI WORD DI BAWAH INI ---
    Ini adalah tempat Anda menempelkan (paste) abstrak dari file Word penulis. 
    Bagian ini akan tampil melebar dan rapi sesuai standar Elsevier. 
    Pastikan Anda tidak menghapus perintah `minipage` di sekitarnya.
    % ---------------------------------------------
  \end{minipage}
  \vspace{6pt}
  \noindent\rule{\textwidth}{0.4pt}
  \vspace{10pt}
\end{strip}

% =========================================================
%  ISI NASKAH (Copy-Paste dari Word per Bab)
% =========================================================

\section{Introduction}
% Paste Pendahuluan dari Word di sini
Ini adalah contoh paragraf pendahuluan. Dalam proses layout, Anda mengambil teks dari Word dan menempelkannya di sini. Jangan lupa untuk menambahkan sitasi dengan perintah seperti ini \citep{Vial2019}.

\section{Literature Review}
% Paste Tinjauan Pustaka di sini
Penelitian terdahulu menunjukkan hasil yang beragam. Menurut \citet{Verhoef2021}, transformasi digital adalah kunci.

\section{Methodology}
% Paste Metodologi di sini
Jelaskan metode penelitian di sini.

\section{Results}
% Paste Hasil di sini
Tuliskan hasil penelitian.

\section{Conclusion}
% Paste Kesimpulan di sini
Tuliskan kesimpulan.

% =========================================================
%  DAFTAR PUSTAKA
% =========================================================
\bibliography{references}

\end{document}
